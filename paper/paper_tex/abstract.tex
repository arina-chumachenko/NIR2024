В исследованиях функциональной нейровизуализации для изучения функций мозга обычно используются подходы, основанные на задачах или состоянии покоя. 
Подходы, основанные на состоянии покоя, обеспечивают гибкость и масштабируемость при оценке функций мозга, в то время как методы, основанные на задачах, обеспечивают качественные возможности локализации. 
Одной из таких моделей является BrainsurfCNN, полностью сверточная модель нейронной сети на основе поверхностного слоя коры головного мозга. 
Существует также другой подход к решению проблемы функционального картирования мозга - это пространственно-ограниченный независимый компонентный анализ (ICA) для функциональной магнитно-резонансной томографии (fMRI), который использует пространственную информацию в рамках ограниченного ICA с обучением по фиксированной точке. 
Основная цель этой работы заключается в решении задачи сегментации функциональных областей fMRI снимком головного мозга, то есть в создании модели, которая принимает в качестве входных данных 4D-тензор характеристик мозга и возвращает карту с необходимыми метками. 
В частности, целью является снижение индексности данных о функциональной связности в состоянии покоя при построении карт активации.
