\section{Заключение}

В этой работе был применён подход для индивидуального прогнозирования контрастов в задачах по функциональным коннектомам с использованием поверхностного CNN.
В ходе экспериментов ранее опубликованная базовая модель достигла более низких значений корреляции, чем средние значения по группе, что может быть связано с моделированием на уровне ROI, при котором не учитывается соответствующий сигнал от остальной части мозга.
Исследование мозга дало прогнозы, которые в целом были сильно коррелированы с результатами исследования МРТ отдельных людей и были весьма специфичны для них.
Также в качестве функции потерь была использована реконструктивно-контрастивную потеря (RC loss), которая значительно улучшила распознаваемость объекта, что соответствует верхней границе теста и повторного тестирования.

Существует несколько расширений существующего подхода, работа над которыми представляет продолжение этого исследования.
Во-первых, можно распространить прогнозы на подкорковые и мозжечковые компоненты мозга.
Во-вторых, повреждение мозга и потеря памяти могут быть применены к другим областям прогнозирования, где важна специфика индивида, например, при индивидуальных траекториях развития заболеваний.
Кроме того, прогнозирование модели BranSurfCNN может быть внедрено в инструменты контроля качества для tfMRI, когда данные повторного тестирования недоступны.

Эксперименты показывают, что нейронная сеть, основанная на работе с поверхность головного мозга человека, может эффективно извлекать полезные многомасштабные характеристики из функциональных коннектомов для прогнозирования контрастов tfMRI, которые очень специфичны для конкретного человека.
