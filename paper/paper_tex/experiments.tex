\section{Вычислительный эксперимент}

\subsection{Датасет}

В этой работе для экспериментов с большим набором данных были использованы предварительно обработанные и предварительно очищенные МРТ в состоянии покоя (3-Tesla rsfMRI) и МРТ для задач (tfMRI) для 1200 испытуемых из проекта Human Connectome Project (HCP, \cite{glasser2013minimal}). 
Сбор и предварительная обработка набора данных были описаны в работах \cite{glasser2013minimal}, \cite{smith2013resting} и \cite{barch2013function}. 
Данные rsfMRI были получены за четыре 15-минутных цикла, каждый из которых включал $1200$ временных точек для каждого испытуемого. 
HCP также реализовал парцелляции на уровне группы, полученные на основе пространственной ICA. 
В этой работе были использованы ROI из $50$-компонентной парцелляции для вычисления функциональных коннектомов. 
Данные tfMRI, полученные HCP, содержат $86$ контрастов из $7$ областей задач (\cite{barch2013function}), а именно: WM (working memory), GAMBLING,
MOTOR, LANGUAGE, SOCIAL RELATIONAL и EMOTION. 
После исследования \cite{tavor2016task} были исключены избыточные отрицательные контрасты, в результате чего было получено $47$ уникальных контрастов. 
Из $1200$ пациентов, прошедших HCP, у $46$ пациентов также были повторные данные 3T-МРТ. 
Включая только испытуемых, у которых были проведены все $4$ rsfMRI-сканирования и $47$ контрастных tfMRI-исследований, набор данных включал $919$ испытуемых для обучения/проверки и $39$ независимых испытуемых (с повторными сканированиями) для оценки.


\subsection{Основной эксперимент}

На рис.\ref{BrSurf3} показана корреляция предсказания моделей с наблюдаемыми картами контрастности того же объекта.
На последующих рисунках показаны предсказанные контрасты заданий, определяемые как те, у которых средняя корреляция между тестом и повторным тестированием по всем испытуемым больше, чем средняя корреляция по всем испытуемым и контрастам.
На рис.\ref{BrSurf4} показана поверхностная визуализация двух контрастов заданий для двух испытуемых.
В то время как средние значения по группе соответствуют грубому характеру индивидуальных контрастов, контрасты, характерные для конкретного объекта, демонстрируют мелкие детали, которые воспроизводятся при повторном тестировании, но стираются в средних показателях по группе.
С другой стороны, предсказания с помощью модели линейной регрессии не учитывали общие особенности активации, характерные для некоторых контрастов.
В целом, прогноз модели BrainSurfCNN неизменно давал самую высокую корреляцию с индивидуальными контрастами tfMRI, приближаясь к верхней границе контрольного показателя повторного тестирования испытуемых.

\begin{figure}
\centering
\includegraphics[scale=0.425]{fig/fig3.png}
\caption{Корреляция между предсказанными и истинными индивидуальными контрастами. LANG, REL, SOC, EMO, WM, и GAMBL являются сокращениями от LANGUAGE, RELATIONAL, SOCIAL, EMOTION, WORKING MEMORY и GAMBLING соответственно.}
\label{BrSurf3}
\end{figure}



\begin{figure}
\centering
\includegraphics[scale=0.55]{fig/fig4.png}
\caption{{Визуализация поверхности для сравнения 2-х заданий у 2-х испытуемых. В крайнем правом столбце приведены средние значения контрастов по группе для сравнения.}}
\label{BrSurf4}
\end{figure}


Матрицы были нормализованы для наглядности, чтобы учесть более высокую вариабельность между истинными и прогнозируемыми контрастами.
Все матрицы имеют диагональное преобладание, что указывает на то, что индивидуальные прогнозы, как правило, наиболее близки к контрастам у одних и тех же испытуемых.

\begin{figure}
\centering
\includegraphics[scale=0.55]{fig/fig5.png}
\caption{{Нормализованные корреляционные матрицы прогнозируемых и истинных контрастов между испытуемыми для трёх заданий и 39 испытуемых.}}
\label{BrSurf5}
\end{figure}


\begin{figure}
\centering
\includegraphics[scale=0.5]{fig/fig6.png}
\caption{{Точность прогнозов для индивида в 23 надежно прогнозируемых задачах.}}
\label{BrSurf6}
\end{figure}

При всех достоверных картах контрастности, карты контрастности, предсказанные моделью BrainSurfCNN, имеют неизменно лучшую точность идентификации объекта по сравнению с моделью линейной регрессии, показанной на рис.\ref{BrSurf5}, и более четкими диагоналями на рис.\ref{BrSurf6}.