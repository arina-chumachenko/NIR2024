\section{Введение}

Функциональная магнитно-резонансная томография (fMRI), основанная на задачах, имеет решающее значение для изучения когнитивных, эмоциональных и двигательных процессов в мозге человека (\cite{anticevic2008 comparing, besle2013single, schaefer2018local}).
Это дает ценную информацию о функциях мозга и позволяет получать биомаркеры для различных поведенческих показателей  (\cite{mcnab2008prefrontal}, \cite{risk2007neural}, \cite{nijhof2015simulating}).
Тем не менее, tfMRI требует тщательного проектирования и обширной предметной подготовки (\cite{church2010task}, \cite{rosazza2018pre}).
В то время как fMRI в состоянии покоя (rsfMRI) завоевала популярность благодаря своей простоте и устойчивости к ошибкам (\cite{power2014studying}, \cite{dubois2016building}).
RsfMRI выявляет крупномасштабные мозговые сети, связанные с различными когнитивными процессами, и может отражать индивидуальные особенности и психологические факторы.
Эта работа сосредоточена на прогнозировании активности мозга, связанной с конкретной задачей, по функциональным связям в состоянии покоя у здоровых индивидов.
Несмотря на методологические различия, tfMRI и rsfMRI используют сходные нейронные процессы, что предполагает предсказуемость.
Важно отметить, что предыдущим моделям линейной регрессии не хватало точности предсказания.
В этой работе будет рассмотрен модель глубокого обучения BrainSurfCNN, использующая современные инструменты для устранения этого недостатка.

Несмотря на то, что машинное обучение продвинулось вперед в различных областях, его применение в исследованиях функциональной магнитно-резонансной томографии (fMRI) отстает, главным образом, из-за ограничений, которые присутствуют у высококачественных наборов данных.
Наборы данных нейровизуализации, как правило, довольно малы и содержат шумы, что создает проблемы для обучения высокопроизводительных нейронных сетей.
Чтобы решить эту проблему, в этой работе представлена BrainSurfCNN -- поверхностная нейронная сеть, которая предназначена для прогнозирования различий в задачах в сравнении с состоянием покоя.
В отличие от предыдущих традиционных подходов, работающих с трехмерными изображениями или двумерными срезами, этот метод использует поверхностное представление, используя геометрию коры головного мозга и сохраняя целостность сигнала.